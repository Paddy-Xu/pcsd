\documentclass[12pt]{article}
\usepackage[utf8]{inputenc}
\usepackage[english]{babel}
\usepackage{listings}
\usepackage{tikz}
\usepackage{amsmath,amssymb}
\tikzset{main node/.style={circle,fill=white!20,draw,minimum size=1cm,inner sep=0pt},}
\usepackage{verbatim}
\newcommand{\HRule}{\rule{\linewidth}{0.5mm}}

\begin{document}

\begin{center}
\textsc{\LARGE Principles of Computer System Design}\\[0.3cm] % Context
\HRule \\[0.4cm]
{ \huge \bfseries Assignment 3} % Main title
\HRule \\[0.4cm]
\large
Johannes de Fine Licht % Names
\\Philip Graae
\\Ola Rønning
\\\today
\end{center}
\section*{Recovery Concepts}
\subsection*{1.}
If a system implements force and no-stealing, the system will not have to implement \texttt{REDO} nor \texttt{UNDO}. The system will not have to implement \texttt{REDO} as the state of system is always consistent on the non-volatile memory, this is because writes are forced to write to the the non-volatile memory. As the system does not implement stealing; non-volatile storage will never have dirty cells, and hence there will be nothing to \texttt{UNDO} after a crash.
\subsection*{2.}
The difference between non-volatile and stable storage, is that stable storage can recover from media failure by implementing redundancy. Both kind of storage can recover from crashes. The final kind of storage, volatile storage, cannot recover from neither media failure nor crashes.
\subsection*{3.}

\section*{ARIES}
\begin{tabular*}{\textwidth}{@{\extracolsep{\fill}}lllll} 
\texttt{LOG}&&&& \\ 
&&&&\\ 
\texttt{LSN} & \texttt{LAST\_LSN} & \texttt{TRAN\_ID} & \texttt{TYPE} & \texttt{PAGE\_ID} \\
\texttt{1} & \texttt{-} & \texttt{-} & \texttt{begin CKPT} & \texttt{-} \\ 
\texttt{2} & \texttt{-} & \texttt{-} & \texttt{end CKPT} & \texttt{-} \\ 
\texttt{3} & \texttt{NULL} & \texttt{T1} & \texttt{update} & \texttt{P2} \\ 
\texttt{4} & \texttt{3} &\texttt{T1} & \texttt{update} & \texttt{P1} \\ 
\texttt{5} & \texttt{NULL} & \texttt{T2} & \texttt{update} & \texttt{P5} \\ 
\texttt{6} & \texttt{NULL} & \texttt{T3} & \texttt{update} & \texttt{P3} \\ 
\texttt{7} & \texttt{6} & \texttt{T3} & \texttt{commit} & \texttt{-} \\ 
\texttt{8} & \texttt{5} & \texttt{T2} & \texttt{update} & \texttt{P5} \\
\texttt{9} & \texttt{8} & \texttt{T2} & \texttt{update} & \texttt{P3} \\
\texttt{10} & \texttt{6} & \texttt{T3} & \texttt{END} & \texttt{-} \\ 
\end{tabular*} 
\section*{Programming Task}
\subsection*{1.}
\subsection*{2.}
\subsection*{3.}
\subsection*{4.}
\subsection*{5.}
\subsection*{6.}
\section*{Discussion on the Performance Measurements}
\subsection*{1.}
\subsection*{2.}
\subsection*{3.}
\end{document}
